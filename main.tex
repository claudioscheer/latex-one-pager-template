\documentclass[a3paper, 11pt]{article}
\usepackage[top=2cm, bottom=2cm, left=2cm, right=2cm]{geometry}
\usepackage{amsmath, amssymb, amsfonts}
\usepackage{graphicx}
\usepackage{hyperref}
\usepackage{fancyhdr}

\pagestyle{fancy}
\fancyhf{}
\fancyhead[L]{Your Name}
\fancyhead[C]{One-Pager Title}
\fancyhead[R]{\thepage}
\fancyfoot[C]{\thepage}

\begin{document}

\title{One-Pager Title}
\author{Your Name}
\date{\today}
\maketitle

\begin{abstract}
	This is a brief abstract summarizing the main points of your document. Keep it concise and to the point.
\end{abstract}

\section{Introduction}
Provide a brief introduction to the topic. Explain the purpose and scope of the document in a few sentences.

\section{Main Content}
\subsection{Subsection 1}
Detail the first main point here. Use bullet points or numbered lists if necessary to organize information clearly.

\subsection{Subsection 2}
Detail the second main point here. Include any relevant equations, figures, or tables.

\begin{equation}
	E = mc^2
\end{equation}

\begin{figure}[h]
	\centering
	\includegraphics[width=0.5\textwidth]{example-image}
	\caption{An example image.}
	\label{fig:example}
\end{figure}

\section{Conclusion}
Summarize the key takeaways from the document. Highlight the main points and any concluding thoughts.

\begin{thebibliography}{9}
	\bibitem{latexcompanion}
	Author Name. \textit{Book Title}. Publisher, Year.
\end{thebibliography}

\end{document}
